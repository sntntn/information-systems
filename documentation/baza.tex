\section{Baza podataka}

Baza podataka informacionog sistema digitalne fitnes platforme projektovana je sa ciljem pouzdanog skladištenja i organizacije svih ključnih podataka neophodnih za funkcionisanje sistema. Sistem je realizovan u mikroservisnoj arhitekturi i primarno koristi NoSQL bazu podataka (MongoDB), što omogućava fleksibilno modelovanje podataka, lako prilagođavanje promenama u domenu i efikasno skaliranje pojedinačnih servisa.

Pored domenskih podataka koji se čuvaju u NoSQL bazama, sistem koristi i relacijsku bazu podataka (Microsoft SQL Server) u okviru IdentityServer komponente. Ova baza namenjena je skladištenju podataka vezanih za autentifikaciju i autorizaciju korisnika, uključujući korisničke naloge, uloge, dozvole i bezbednosne tokene. Upotreba relacione baze u ovom delu sistema obezbeđuje visok nivo
konzistentnosti i integriteta podataka prilikom upravljanja identitetima, dok se u ostalim mikroservisima zadržavaju prednosti dokumentno-orijentisanog pristupa.

Podaci u NoSQL bazama organizovani su u zasebne kolekcije koje odgovaraju osnovnim domenskim entitetima sistema, kao što su korisnici (klijenti i treneri), treninzi, video treninzi, planovi ishrane, rezervacije termina, čet sesije, notifikacije i platne transakcije. Svaki mikroservis poseduje sopstvenu bazu podataka, čime se obezbeđuje jasna razdvojenost odgovornosti i nezavisnost servisa.

Veze između entiteta ostvarene su referencama putem identifikatora (npr. TrainerId, ClientId, ClientIds), umesto korišćenja klasičnih relacijskih stranih ključeva. Ovakav pristup je u skladu sa principima NoSQL baza podataka i mikroservisne arhitekture. Na primer, entitet trening sadrži listu identifikatora klijenata koji učestvuju u treningu, dok se poruke u okviru četa čuvaju kao ugnežđeni dokumenti
unutar odgovarajuće čet sesije.

Ovakav dizajn baze podataka omogućava jasno razdvajanje korisničkih uloga i njihovih aktivnosti, efikasno praćenje veza između trenera, klijenata i sadržaja, kao i jednostavno izdvajanje podataka potrebnih za analitičke servise. Istovremeno se obezbeđuje
pouzdano i stabilno funkcionisanje celokupnog informacionog sistema.

\begin{figure}[H]
    \centering
    \includegraphics[width=\textwidth]{diagrams/data_base_diagrams/Data class diagram.png}
    \caption{UML dijagram klasa baze podataka}
\end{figure}
