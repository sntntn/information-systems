\section{Zaključak}


Razvoj informacionog sistema digitalne fitnes platforme predstavlja savremeno softversko rešenje koje podržava moderan način vežbanja i vođenja zdravog načina života. Projektovani sistem objedinjuje upravljanje treninzima, zakazivanje termina, izradu i praćenje planova ishrane, komunikaciju između klijenata i trenera, obradu online plaćanja i distribuciju obaveštenja u okviru jedinstvene platforme.

Implementirano rešenje omogućava efikasno povezivanje klijenata, trenera i administratora, čime se unapređuje organizacija rada i kvalitet pruženih usluga, uz smanjenje administrativnog opterećenja i bolji pregled nad funkcionisanjem sistema.

Poseban značaj sistema ogleda se u unapređenju korisničkog iskustva. Klijentima je omogućen personalizovan pristup treniranju i planovima ishrane, kontinuirana komunikacija sa trenerima putem četa, kao i praćenje sopstvenog napretka kroz individualne, grupne i video treninge, što pozitivno utiče na motivaciju i postizanje željenih rezultata. Istovremeno, trenerima se obezbeđuje lakša organizacija treninga kroz efikasno upravljanje rasporedima, kao i centralizovan i detaljan uvid u napredak i individualne potrebe klijenata, čime se unapređuje kvalitet planiranja i realizacije treninga.

Arhitektura sistema zasnovana je na mikroservisnom pristupu i projektovana je modularno, što omogućava dalji razvoj i proširivanje funkcionalnosti. Kao potencijalna unapređenja izdvajaju se razvoj mobilne aplikacije, naprednije analitičke funkcionalnosti, kao i integracija inteligentnih asistivnih sistema zasnovanih na veštačkoj inteligenciji koji bi korisnicima pružali personalizovane preporuke i podršku pri korišćenju platforme.

Na ovaj način, digitalna fitnes platforma predstavlja stabilnu i skalabilnu osnovu za digitalizaciju procesa u oblasti fitnesa i zdravog načina života, uz mogućnost kontinuiranog prilagođavanja savremenim potrebama korisnika i daljeg unapređenja kvaliteta usluga.
\newpage