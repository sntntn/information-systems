\section{Uvod}

Informacioni sistem za upravljanje treninzima predstavlja softversko rešenje namenjeno digitalizaciji i unapređenju procesa planiranja, realizacije i praćenja fizičkih treninga. Sistem omogućava interakciju između klijenata i trenera kroz rezervaciju grupnih i individualnih treninga, korišćenje video treninga, razmenu poruka putem četa, kao i online plaćanje usluga. Cilj sistema je da obezbedi centralizovanu platformu koja povećava efikasnost rada trenera, poboljšava korisničko iskustvo klijenata i omogućava jednostavan pristup treninzima bez obzira na vremenska i prostorna ograničenja.

Sistem integriše ključne funkcionalnosti kao što su upravljanje korisničkim nalozima, kreiranje i korišćenje video treninga, zakazivanje termina, obrada plaćanja i upravljanje obaveštenjima. Na taj način omogućava transparentan i pregledan tok informacija između svih učesnika, uz jasno definisane uloge i odgovornosti.

\subsection{Akteri}

Sistem prepoznaje sledeće grupe korisnika koji interaguju sa njegovim funkcionalnostima na različite načine:

\begin{itemize}
    \item \textbf{Klijent}: Registrovani korisnik sistema koji koristi usluge treninga. Klijent može pregledati dostupne treninge, rezervisati individualne i grupne treninge, kupovati i koristiti video treninge, komunicirati sa trenerima putem četa, vršiti online plaćanja i upravljati svojim obaveštenjima.
    
    \item \textbf{Trener}: Korisnik sistema zadužen za kreiranje i upravljanje treninzima. Trener može kreirati video vežbe i video treninge, definisati strukturu treninga, upravljati terminima, komunicirati sa klijentima i pratiti realizaciju treninga.

    \item \textbf{Administrator}: Korisnik sistema zadužen za tehničku administraciju i održavanje sistema. Administrator može dodavati ili uklanjati korisnike, upravljati pravima pristupa, pratiti i rešavati tehničke probleme, kao i nadgledati rad sistema u celini.
    
    \item \textbf{Platni sistem}: Eksterni sistem koji obrađuje online plaćanja. Odgovoran je za validaciju i realizaciju transakcija prilikom kupovine treninga ili rezervacije termina.
\end{itemize}

\subsection{Korišćeni alati}

Za analizu, projektovanje i dokumentovanje informacionog sistema korišćen je skup standardizovanih jezika za modelovanje i odgovarajući softverski alati:

\begin{itemize}
    \item \textbf{UML (Unified Modeling Language)} dijagrami:
    \begin{itemize}
        \item \textbf{Slučajevi upotrebe}: Za opis funkcionalnosti sistema iz perspektive korisnika.
        \item \textbf{Dijagrami sekvence}: Za prikaz interakcije između aktera i sistemskih komponenti tokom realizacije slučajeva upotrebe.
        \item \textbf{Dijagrami aktivnosti}: Za modelovanje toka procesa i korisničkih radnji.
    \end{itemize}
    
    \item \textbf{BPMN (Business Process Model and Notation)}: Za modelovanje poslovnih procesa, naročito u delu obrade plaćanja i rezervacija.
    
    \item \textbf{LaTeX}: Korišćen za izradu i formatiranje tehničke dokumentacije, zbog podrške za strukturiranje sadržaja i profesionalni izgled dokumenata.
\end{itemize}
\newpage