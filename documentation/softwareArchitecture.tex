\section{Arhitektura Sistema}

\subsection{Korišćena Arhitektura Sistema}

    Aplikacija FitPlusPlus razvijena je korišćenjem mikroservisne arhitekture, u okviru koje je funkcionalnost sistema podeljena na skup nezavisnih, slabo povezanih servisa. Svaki mikroservis predstavlja samostalnu celinu sa jasno definisanom odgovornošću, sopstvenom bazom podataka i jasno definisanim interfejsima za komunikaciju sa ostalim delovima sistema.

    Ovakav arhitektonski pristup izabran je zbog sledećih prednosti:
    \begin{itemize}
        \item lakšeg održavanja i razvoja sistema,
        \item mogućnosti nezavisnog skaliranja pojedinačnih servisa,
        \item veće robusnosti sistema u slučaju grešaka,
        \item jasnog razdvajanja domena i odgovornosti.
    \end{itemize}

    Sistem se sastoji od frontend aplikacije i niza backend mikroservisa, pri čemu frontend komunicira sa backendom isključivo preko API
    Gateway komponente. API Gateway predstavlja centralnu ulaznu tačku u sistem i odgovoran je za rutiranje zahteva ka odgovarajućim mikroservisima, kao i za objedinjavanje komunikacije prema klijentskoj aplikaciji.

    Za otkrivanje i registraciju servisa koristi se Service Discovery mehanizam (Consul), koji omogućava dinamičko pronalaženje dostupnih instanci mikroservisa bez potrebe za statičkom konfiguracijom adresa.

    Backend mikroservisi komuniciraju međusobno koristeći kombinaciju:
    \begin{itemize}
        \item REST API-ja za sinhronu komunikaciju,
        \item gRPC protokola za efikasnu internu komunikaciju između servisa,
        \item asinhrone razmene poruka putem RabbitMQ-a, što omogućava asinhrono povezivanje servisa i veću otpornost sistema.
    \end{itemize}

    Za potrebe rada u realnom vremenu, kao što je direktna čet komunikacija između klijenata i trenera, koristi se WebSocket protokol, koji omogućava dvosmernu i kontinuiranu razmenu poruka bez potrebe za ponovnim slanjem HTTP zahteva.

    Svaki mikroservis poseduje sopstvenu bazu podataka, čime se primenjuje princip \textit{database-per-service}.

    Kompletan sistem je kontejnerizovan korišćenjem Docker-a, dok je orkestracija servisa realizovana pomoću Docker Compose-a, što  omogućava jednostavno pokretanje, konfiguraciju i razvoj sistema u različitim okruženjima.

    
\subsubsection{Frontend aplikacija}

Osnovna uloga frontend aplikacije je omogućavanje interakcije korisnika sa sistemom kroz intuitivan i responzivan korisnički interfejs. Ne sadrži poslovnu logiku sistema, već služi kao posrednik između korisnika i backend mikroservisa, sa kojima komunicira putem API Gateway komponente.

Pristup funkcionalnostima aplikacije prilagođen je korisničkim ulogama, pri čemu se nakon uspešne autentifikacije korisnicima prikazuju odgovarajući pogledi i opcije u skladu sa njihovim privilegijama (klijent, trener ili administrator). \\
\\
\textbf{Komponente:}
\begin{enumerate}
    \item \textbf{Administratorska perspektiva} \\
    Omogućava pregled i upravljanje osnovnim podacima o korisnicima sistema (klijentima i trenerima).

    \item \textbf{Perspektiva trenera} \\
    Omogućava:
    \begin{itemize}
        \item upravljanje individualnim i grupnim treninzima,
        \item definisanje dostupnih termina,
        \item kreiranje i upravljanje nutritivnim planovima,
        \item komunikaciju sa klijentima putem čet sistema u realnom vremenu,
        \item dodavanje i upravljanje video treninzima,
        \item pregled analitičkih podataka o treninzima i klijentima.
    \end{itemize}

    \item \textbf{Perspektiva klijenta} \\
    Omogućava:
    \begin{itemize}
        \item rezervaciju individualnih i grupnih treninga,
        \item pregled i praćenje nutritivnih planova,
        \item komunikaciju sa trenerima putem četa,
        \item pristup kupljenim video treninzima,
        \item pregled lične statistike i analitike korišćenja aplikacije.
    \end{itemize}
\end{enumerate}

\textbf{Tehnologije:}
\begin{itemize}
    \item JavaScript (ES6+), HTML5, CSS3
    \item Frontend aplikacija koristi REST API i WebSocket komunikaciju za interakciju sa backend servisima
\end{itemize}

\textbf{Bezbednost UI sloja obezbeđena je primenom sledećih mehanizama:}
\begin{itemize}
    \item komunikacija sa backend servisima putem HTTPS protokola,
    \item autentifikacija zasnovana na JWT tokenima dobijenim od IdentityServer komponente,
    \item autorizacija zasnovana na korisničkim ulogama i permisijama, pri čemu se pristup funkcionalnostima dinamički prilagođava u zavisnosti od uloge korisnika.
\end{itemize}

\subsubsection{Backend aplikacija}

Backend aplikacija predstavlja centralni deo sistema u kome je implementirana poslovna logika digitalne fitnes platforme. Backend je realizovan kao skup nezavisnih mikroservisa, od kojih svaki obrađuje tačno definisan deo domenskog problema i implementira odgovarajuća poslovna pravila.

Mikroservisi u okviru backend aplikacije:
\begin{itemize}
    \item obrađuju zahteve pristigle sa frontend aplikacije putem API Gateway komponente,
    \item sprovode validaciju podataka i primenu poslovnih pravila,
    \item upravljaju sopstvenim skupovima podataka,
    \item razmenjuju podatke sa ostalim mikroservisima korišćenjem sinhrone i asinhrone komunikacije.
\end{itemize}

U skladu sa mikroservisnom arhitekturom, primenjen je princip \textit{database-per-service}, gde svaki mikroservis poseduje sopstvenu bazu podataka i u potpunosti je odgovoran za upravljanje svojim podacima.

\textbf{Mikroservisi:}
\begin{itemize}
    \item \textbf{Client Service} \\
    Zadužen za upravljanje podacima o klijentima, njihovim profilima i povezanim informacijama.

    \item \textbf{Trainer Service} \\
    Omogućava upravljanje profilima trenera, dostupnim treninzima i povezanim podacima.

    \item \textbf{Reservation Service} \\
    Omogućava rezervaciju individualnih i grupnih treninga, upravljanje terminima, otkazivanje i praćenje dostupnosti. Servis je integrisan sa Payment, Notification i Analytics servisima.

    \item \textbf{Payment Service} \\
    Zadužen za obradu plaćanja treninga i dodatnih usluga, kao i za evidenciju platnih transakcija.

    \item \textbf{Chat Service} \\
    Omogućava realizaciju online mentorstva između klijenata i trenera putem direktne komunikacije u realnom vremenu, korišćenjem WebSocket tehnologije.

    \item \textbf{Video Training Service} \\
    Omogućava upravljanje video treninzima i vežbama. Treneri mogu kreirati i održavati video sadržaj, dok klijenti imaju pristup kupljenim treninzima.

    \item \textbf{Review Service} \\
    Omogućava klijentima i trenerima ostavljanje ocena i komentara nakon završenih treninga.

    \item \textbf{Nutrition Service} \\
    Zadužen za upravljanje nutritivnim planovima i ciljevima ishrane. Treneri mogu kreirati planove ishrane, dok klijenti mogu pratiti unos kalorija i lične ciljeve.

    \item \textbf{Notification Service} \\
    Zadužen za slanje sistemskih obaveštenja korisnicima u vezi sa rezervacijama, plaćanjima i drugim relevantnim događajima.

    \item \textbf{Analytics Service} \\
    Prikuplja i obrađuje podatke o aktivnostima korisnika i treninga, omogućavajući generisanje statistike i analitičkih pregleda.
\end{itemize}

\textbf{Tehnologije i komunikacija}
\begin{itemize}
    \item \textbf{Backend tehnologija}: ASP.NET Core (C\#)
    \item \textbf{Sinhrona komunikacija}: REST API i gRPC
    \item \textbf{Asinhrona komunikacija}: RabbitMQ (Event Bus)
    \item \textbf{Real-time komunikacija}: WebSockets
\end{itemize}

\begin{figure}[H]
    \centering
    \includegraphics[width=0.5\linewidth]{images/architectureDiagram.png}
    \caption{Dijagram Arhitekture Softvera}
\end{figure}
