\section{Arhitektura Sistema}

\subsection{Korišćena Arhitektura Sistema}

    U razvoju aplikacije FitPlusPlus korišćena je \textbf{višeslojna arhitektura} (3-tier architecture) gde svaki deo je izdeljen na mikroservise. Ovaj način organizacije je izabran jer omogućava lakše održavanje i jednostavniju nadogradnju sistema novim funkcionalnostima. Razvijena je web-aplikacija sa 3 osnovna pogleda, iz perspektive \textit{korisnika}, \textit{trenera} i \textit{administratora}. 
    
\subsubsection{UI Layer}

\begin{description}
  \item[\textbf{Opis}] 
  Predstavlja vizualni interfejs za korišćenje aplikacije, kao i interakcije korisnika i trenera, i upravljanje sistemom kao administrator.

  \item[\textbf{Komponente}]~
  
  \begin{itemize}
    \item \textbf{Admin Perspektiva} – Ova perspektiva ima meni sa listama svih klijenata i svih trenera.
    \item \textbf{Trener Perspektiva} – Ova perspektiva ima meni sa opcijama za: određivanje slobodnih termina za individualni trening, održavanje grupnog treninga, postavljanje nutricionih planova, chat sa klijentima, dodavanje video treninga i analitike korišćenja aplikacije.
    \item \textbf{Klijent Perspektiva} – Ova perspektiva ima meni sa opcijama za: rezervaciju individualnog treninga, rezervaciju grupnog treninga, nutricione planove, chat, video treninge kao i analitike korišćenja aplikacije.
  \end{itemize}

  \item[\textbf{Tehnologije}]
  Vue.js, API pozivi ka dockerizovanom backendu
\item[\textbf{Bezbednost}:] HTTPS, Autentikacija(JWT Tokeni), Permisije zasnovane na ulogama 
\end{description}



\subsubsection{Buisness Logic Layer}
\begin{description}
    \item[\textbf{Opis}:] Predstavlja sloj na kojem se nalazi svi zahtevi, validacije, zaštite, pravila rada aplikacije objasnjeni kroz mikroservise.
    
    \item[\textbf{Mikroservisi}:] ~
        \begin{itemize}
            \item \textbf{Identity Service} - Servis koji je zadužen za autentikaciju i autorizaciju korisnika.
            \item \textbf{Client Service} - Servis koji je zadužen za za upravljanje profila klijenta, upravljanje profila i podacima od članstvu. 
            \item \textbf{Trainer Service} - Servis koji je zadužen za upravljanje profila trenera, rasporeda i istorije treninga.
            \item \textbf{Review Service} - Servis koji je dozvoljava klijentima i trenerima da ostave ocene treingna
            \item \textbf{Payment Service} - Servis zaduzen za obradu placanja treninga.
            \item \textbf{Chat Service} - Servis koji je zadužen za direktnu komunikaciju između klijenta i trenera, i omogućava treneru da usluži online mentorstvo.
            \item \textbf{Video Training Service} - Servis koji omogucava korisnicima pristup biblioteci sa snimcima treninga, i informacije trenera. Treneri dodanto mogu da ubacuju ili brisu snimke.
            \item \textbf{Reservation Service} - Servis koji omogućava rezervisanje grupnih ili individualnih treninga, zakazivanje, otakzivanje i praćenje dostupnosti termina u realnom vremenu.
            \item \textbf{Notification Service} - Servis zadužen za push i email notifikacije, vezan za sve rezervacije, plaćanja.
            \item \textbf{Analitics Service} - Servis koji omogućava korisnicima da prate svoje statistike korišćenja FitPlusPlus aplukacije.
            \item \textbf{Nutrition  Service} - Servis koji je zadužen za upravljanjem planova ishrane i praćenja kalorija. Trener može da dodaje hrane i njihove nutritivne vrednosti, da pravi planove za sve tipove cilja, da dodaje, briše i menja trenutne planove.
            \item \textbf{Gateway and Discovery Service} - Servis koji služi kao centralizovani "gateway" i stara se o tome da svaki poziv ode odgovarajućem mikorservisu. 
        \end{itemize}
    \item[\textbf{Tehnologije}:] ASP.NET Core, C\#, RabbitMQ, Websockets, GRPC, REST API, Ocelot, Consul 
    \item[\textbf{Integracije}:] FluentEmail, Paypal
\end{description}

\subsubsection{Database Layer}
\begin{description}
    \item[\textbf{Opis}:] Sloj koji opisuje način čuvanja i prisutpa podacima.
    \item[\textbf{Komponente}:]~
         \begin{itemize}
            \item Baza podataka - Koriste se baze podataka za skoro svaki mikro servis gde je glavna nerelaciona baza podataka MongoDB.\newline
            Kljucni entiti su: Training, Excercise, Trainer, Client.
        \end{itemize} 
    \item[\textbf{Bezbednost}:] Pristupanje na osnovu uloga, enkripcija podataka
\end{description}

\begin{figure}[h!]
    \centering
    \includegraphics[width=0.5\linewidth]{images/architectureDiagram.png}
    \caption{Dijgram Arhitekture Softwera}
\end{figure}
