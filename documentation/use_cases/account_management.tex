\subsection{Slučaj upotrebe: Upravljanje registrovanim nalozima}

\subsubsection*{Kratak opis}
Administrator upravlja registrovanim korisničkim nalozima u sistemu. Sistem omogućava dodavanje mejl adresa budućih trenera, izmenu informacija o trenerima, kao i brisanje klijentskih i trenerskih naloga uz proveru postojećih zakazanih i plaćenih usluga.

\subsubsection*{Akteri}
\begin{itemize}
    \item \textbf{Administrator} -- korisnik sistema zadužen za upravljanje korisničkim nalozima.
\end{itemize}

\subsubsection*{Preduslovi}
\begin{itemize}
    \item Administrator je prijavljen u sistem.
\end{itemize}

\subsubsection*{Osnovni tok}
\begin{enumerate}
    \item Administrator pristupa administratorskoj sekciji za upravljanje korisničkim nalozima.
    \item Sistem prikazuje listu registrovanih korisnika i dostupne administrativne akcije.
    \item Tok se nastavlja izvršavanjem podtokova:
    \begin{itemize}
        \item \textbf{P1: Dodavanje mejl adrese trenera}
        \item \textbf{P2: Izmena informacija o treneru}
        \item \textbf{P3: Brisanje klijentskog naloga}
        \item \textbf{P4: Brisanje trenerskog naloga}
    \end{itemize}
    \item Administrator završava upravljanje korisničkim nalozima i napušta administratorsku sekciju.
\end{enumerate}

\subsubsection*{Postuslovi}
\begin{itemize}
    \item Izmene nad korisničkim nalozima su sačuvane u sistemu.
    \item Statusi korisničkih naloga su ažurirani u skladu sa izvršenim akcijama administratora.
\end{itemize}

\subsubsection*{Podtokovi}

\textbf{P1: Dodavanje mejl adrese trenera}
\begin{enumerate}
    \item Administrator unosi mejl adresu korisnika koji treba da postane trener.
    \item Sistem proverava da li je nalog sa tom mejl adresom već registrovan.
    \item Ako nalog nije registrovan, sistem dodaje mejl adresu na listu trenera.
    \item Kada se korisnik registruje sa tom mejl adresom, sistem mu automatski dodeljuje ulogu trenera.
\end{enumerate}

\textbf{P2: Izmena informacija o treneru}
\begin{enumerate}
    \item Administrator bira trenerski nalog iz liste korisnika.
    \item Sistem prikazuje trenutne informacije o treneru.
    \item Administrator menja opšte informacije o treneru (npr. biografiju).
    \item Administrator potvrđuje izmene.
    \item Sistem čuva ažurirane informacije o treneru.
\end{enumerate}

\textbf{P3: Brisanje klijentskog naloga}
\begin{enumerate}
    \item Administrator bira klijentski nalog koji želi da obriše.
    \item Sistem proverava da li klijent ima zakazane i plaćene usluge koje još nisu iskorišćene (individualni treninzi ili mentorski čet).
    \item Ako takve usluge ne postoje, sistem briše klijentski nalog.
    \item Ako je klijent bio prijavljen na grupni trening, sistem automatski otkazuje učešće klijenta i oslobađa mesto za druge klijente.
\end{enumerate}

\textbf{P4: Brisanje trenerskog naloga}
\begin{enumerate}
    \item Administrator bira trenerski nalog koji želi da obriše.
    \item Sistem proverava da li trener ima zakazane i plaćene usluge (individualne treninge ili mentorski čet).
    \item Ako takve usluge ne postoje, sistem briše trenerski nalog.
\end{enumerate}

\subsubsection*{Alternativni tokovi}

\textbf{A1: Brisanje naloga nije dozvoljeno}
\begin{itemize}
    \item Tokom podtokova \textbf{P3} ili \textbf{P4} sistem utvrđuje da korisnički nalog ne može biti obrisan zbog postojećih zakazanih i plaćenih usluga.
    \item Sistem prikazuje administratoru jasnu poruku sa razlogom zbog kog brisanje nije moguće.
    \item Podtok se završava bez brisanja korisničkog naloga.
\end{itemize}

\subsubsection*{Dodatne informacije}
\begin{itemize}
    \item Dodavanje mejl adrese na listu trenera je jedini način na koji korisnik može dobiti ulogu trenera u sistemu.
    \item Brisanjem korisničkog naloga uklanjaju se svi povezani podaci, osim onih koje je sistem dužan da zadrži iz zakonskih ili evidencionih razloga.
    \item Sistem obezbeđuje integritet podataka prilikom brisanja naloga i povezanih rezervacija.
\end{itemize}

\begin{figure}[h!]
    \centering
    \includegraphics[width=0.9\linewidth]{images/upravljanje_nalozima.png}
    \caption{Dijagram slučaja upotrebe "Upravljanje registrovanim nalozima"}
\end{figure}

\begin{figure}[h!]
    \centering
    \includegraphics[width=0.9\linewidth]{diagrams/activity_diagrams/Registration_activity_diagram.pdf}
    \caption{Dijagram aktivnosti "Upravljanje nalozima"}
\end{figure}

\begin{figure}[h!]
    \centering
    \includegraphics[width=0.9\linewidth]{diagrams/state_diagrams/manage_accounts.png}
    \caption{Dijagram stanja "Upravljanje nalozima"}
\end{figure}


\FloatBarrier