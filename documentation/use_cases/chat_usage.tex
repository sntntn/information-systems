\subsection{Slučaj upotrebe: Korišćenje četa}

\subsubsection*{Kratak opis}
Sistem omogućava razmenu poruka klijenta i odabranog trenera u real-time režimu tokom perioda od 30 dana od trenutka zakazivanja mentorskog četa. Nakon isteka tog perioda, poruke ostaju dostupne za čitanje, ali slanje novih poruka nije dozvoljeno.

\subsubsection*{Akteri}
\begin{itemize}
    \item \textbf{Klijent} -- korisnik koji komunicira sa trenerom u okviru kupljenog mentorskog četa.
    \item \textbf{Trener} -- korisnik koji komunicira sa klijentom u okviru mentorskog četa.
\end{itemize}

\subsubsection*{Preduslovi}
\begin{itemize}
    \item Mentorski čet je uspešno zakazan (slučaj upotrebe \textbf{"Rezervisanje čet mentorstva"} je završen uspešno).
    \item Klijenti trener moraju biti registrovani.
    \item Nije istekao period od 30 dana od trenutka zakazivanja mentorskog četa.
\end{itemize}

\subsubsection*{Osnovni tok}
\begin{enumerate}
    \item Klijent ili trener se uloguje u aplikaciju.
    \item Klijent ili trener otvara mentorski čet.
    \item Sistem proverava da li je mentorski čet aktivan (da li je prošlo manje od 30 dana od zakazivanja).
    \item Sistem prikazuje istoriju poruka i trenutni status mentorskog četa.
    \item Tok se nastavlja izvršavanjem podtokova \textbf{Slanje poruke} i \textbf{Prijem i prikaz poruke} sve dok:
    \begin{itemize}
        \item mentorski čet ne istekne ili
        \item korisnik ne napusti razgovor
    \end{itemize}
    \item Korisnik napušta mentorski čet.
\end{enumerate}

\subsubsection*{Postuslovi}
\begin{itemize}
    \item Sve razmenjene poruke su trajno sačuvane u sistemu.
    \item Sistem je evidentirao autora i vreme slanja svake poruke.
    \item Poruke su dostupne za kasnije čitanje i pregled istorije razgovora.
\end{itemize}

\subsubsection*{Podtokovi}

\textbf{P1: Slanje poruke}
\begin{enumerate}
    \item Klijent ili trener unosi tekst poruke i bira opciju \textbf{Pošalji}.
    \item Sistem proverava da li je poruka validna (npr. nije prazna).
    \item Sistem čuva poruku zajedno sa informacijama o pošiljaocu i vremenu slanja.
    \item Sistem prikazuje poslatu poruku pošiljaocu.
    \item Sistem isporučuje poruku drugom učesniku.
\end{enumerate}

\textbf{P2: Prijem i prikaz poruke}
\begin{enumerate}
    \item Sistem prima novu poruku namenjenu korisniku.
    \item Ako je mentorski čet trenutno otvoren, sistem odmah prikazuje poruku korisniku.
    \item Ako mentorski čet nije otvoren, sistem evidentira novu poruku i može poslati notifikaciju korisniku.
\end{enumerate}

\subsubsection*{Alternativni tokovi}

\textbf{A1: Mentorski čet je istekao (read-only režim)}
\begin{itemize}
    \item U koraku 2 sistem utvrđuje da je istekao period od 30 dana od zakazivanja.
    \item Sistem prikazuje istoriju poruka, ali onemogućava unos i slanje novih poruka.
    \item Ako korisnik pokuša da pošalje poruku, sistem prikazuje obaveštenje da je mentorski čet istekao.
    \item Slučaj upotrebe se nastavlja samo u režimu čitanja ili se završava.
\end{itemize}

\textbf{A2: Slanje poruke nije uspešno}
\begin{itemize}
    \item U koraku 6 dolazi do sistemske greške (npr. problem sa bazom ili mrežom).
    \item Sistem obaveštava korisnika da poruka nije poslata.
    \item Korisnik može pokušati ponovo (povratak na korak 4) ili odustati, čime se slučaj upotrebe završava.
\end{itemize}

\subsubsection*{Dodatne informacije}
\begin{itemize}
    \item Mentorski čet funkcioniše u real-time režimu kako bi nove poruke bile odmah prikazane korisnicima.
    \item Inicijalne predefinisane poruke mogu biti automatski poslate u ime trenera odmah nakon uspešne uplate i zakazivanja mentorskog četa, kao deo prethodnog slučaja upotrebe.
    \item Nakon isteka mentorskog perioda, čet prelazi u \textbf{read-only} režim, ali se kompletna istorija poruka trajno čuva u sistemu.
    \item Čet u aplikaciji služi isključivo za razmenu poruka između klijenta i odabranog trenera. Trener - trener ili klijent - klijent četovi nisu podržani.
\end{itemize}

\begin{figure}[h!]
    \centering
    \includegraphics[width=0.9\linewidth]{images/koristi_cet.png}
    \caption{Dijagram slučaja upotrebe "Korišćenje četa"}
\end{figure}


\begin{figure}[h!]
    \centering
    \includegraphics[width=0.9\linewidth]{diagrams/activity_diagrams/Chat_activity_diagram.pdf}
    \caption{Dijagram aktivnosti "Korišćenje četa"}
\end{figure}





\FloatBarrier