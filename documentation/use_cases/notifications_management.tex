\subsection{Slučaj upotrebe: Upravljanje obaveštenjima}

\subsubsection*{Kratak opis}
Korisnik upravlja obaveštenjima koja dobija kao rezultat različitih aktivnosti u sistemu. Sistem omogućava pregled liste obaveštenja, pregled detalja obaveštenja, označavanje obaveštenja kao pročitanih i brisanje obaveštenja.

\subsubsection*{Akteri}
\begin{itemize}
    \item \textbf{Korisnik} -- registrovani korisnik sistema koji prima i upravlja svojim obaveštenjima.
\end{itemize}

\subsubsection*{Preduslovi}
\begin{itemize}
    \item Korisnik mora da bude registrovan.
    \item Postoje generisana obaveštenja za korisnika kao rezultat drugih slučajeva upotrebe sistema.
\end{itemize}

\subsubsection*{Osnovni tok}
\begin{enumerate}
    \item Korisnik se uloguje u aplikaciju.
    \item Korisnik otvara listu sa obaveštenjima u aplikaciji.
    \item Sistem prikazuje listu obaveštenja, pri čemu su obaveštenja inicijalno prikazana samo po naslovima i sa jasno označenim statusom (pročitano / nepročitano).
    \item Tok se nastavlja izvršavanjem podtokova \textbf{P1: Pregled obaveštenja} i \textbf{P2: Brisanje obaveštenja} sve dok korisnik ne odluči da zatvori listu obaveštenja.
    \item Korisnik zatvara listu obaveštenja.
\end{enumerate}

\subsubsection*{Postuslovi}
\begin{itemize}
    \item Statusi obaveštenja su ažurirani u skladu sa akcijama korisnika.
    \item Obrisana obaveštenja više nisu prikazana u listi obaveštenja.
\end{itemize}

\subsubsection*{Podtokovi}

\textbf{P1: Pregled obaveštenja}
\begin{enumerate}
    \item Korisnik bira obaveštenje iz liste (pročitano ili nepročitano).
    \item Sistem prikazuje detalje izabranog obaveštenja.
    \item Ako je izabrano obaveštenje bilo u statusu \textbf{nepročitano}, sistem automatski menja njegov status u \textbf{pročitano}.
    \item Korisnik se vraća na listu obaveštenja.
\end{enumerate}

\textbf{P2: Brisanje obaveštenja}
\begin{enumerate}
    \item Korisnik bira jedno ili više obaveštenja za brisanje.
    \item Korisnik potvrđuje brisanje.
    \item Sistem uklanja izabrana obaveštenja iz liste.
\end{enumerate}

\subsubsection*{Alternativni tokovi}

\textbf{A1: Nema dostupnih obaveštenja}
\begin{itemize}
    \item U koraku 2 osnovnog toka sistem ne pronalazi nijedno obaveštenje za korisnika.
    \item Sistem prikazuje poruku da nema dostupnih obaveštenja.
    \item Slučaj upotrebe se završava.
\end{itemize}

\subsubsection*{Dodatne informacije}
\begin{itemize}
    \item Obaveštenja mogu nastati kao posledica različitih slučajeva upotrebe sistema (npr. zakazivanje mentorskog četa, nova poruka u četu, promene statusa treninga).
    \item Korisnik može otvoriti detalje i pročitanih i nepročitanih obaveštenja; samo u slučaju nepročitanih obaveštenja sistem menja njihov status u \textbf{pročitano}.
    \item Brisanjem obaveštenja uklanja se samo korisnička kopija obaveštenja, dok sistemska evidencija može ostati sačuvana.
\end{itemize}

\begin{figure}[h!]
    \centering
    \includegraphics[width=0.9\linewidth]{images/upravljanje_obavestenjima.png}
    \caption{Dijagram slučaja upotrebe "Upravljanje obaveštenjima"}
\end{figure}


\begin{figure}[h!]
    \centering
    \includegraphics[width=0.9\linewidth]{diagrams/sequence_diagrams/manage_notifications.png}
    \caption{Sekvencijalni dijagram "Upravljanje obaveštenjima"}
\end{figure}

\begin{figure}[h!]
    \centering
    \includegraphics[width=0.9\linewidth]{diagrams/state_diagrams/manage_notifications.png}
    \caption{Dijagram stanja "Upravljanje obaveštenjima"}
\end{figure}

\FloatBarrier