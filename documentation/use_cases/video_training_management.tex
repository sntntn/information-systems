\subsection{Slučaj upotrebe: Upravljanje video treninzima}

\subsubsection*{Kratak opis}
Trener upravlja sopstvenim video vežbama i video treninzima. Sistem omogućava treneru da kreira, briše i pregleda vežbe i treninge, kao i da kombinuje vežbe u treninge sa definisanim brojem ponavljanja i setova. Trener može izvršavati ove aktivnosti proizvoljan broj puta tokom korišćenja sistema.

\subsubsection*{Akteri}
\begin{itemize}
    \item \textbf{Trener} -- korisnik koji kreira i upravlja sopstvenim video vežbama i treninzima.
\end{itemize}

\subsubsection*{Preduslovi}
\begin{itemize}
    \item Trener mora da bude registrovan.
\end{itemize}

\subsubsection*{Osnovni tok}
\begin{enumerate}
    \item Trener se uloguje u aplikaciju.
    \item Trener otvara tab za "Video trainings" u navigaciji.
    \item Sistem prikazuje listu vežbi koje je trener kreirao i listu treninga koje je trener kreirao.
    \item Trener bira jednu od dostupnih akcija:
    \begin{itemize}
        \item kreiranje nove vežbe,
        \item kreiranje novog treninga,
        \item brisanje postojeće vežbe,
        \item brisanje postojećeg treninga.
    \end{itemize}
    \item U zavisnosti od izbora, sistem izvršava odgovarajući podtok.
    \item Trener se vraća na stranicu za upravljanje video treninzima.
    \item Koraci 3--5 se ponavljaju sve dok trener ne odluči da napusti stranicu.
\end{enumerate}

\subsubsection*{Postuslovi}
\begin{itemize}
    \item Vežbe i treninzi su ažurirani u skladu sa akcijama koje je trener izvršio.
    \item Sistem je sačuvao sve validne izmene koje je trener napravio.
\end{itemize}

\subsubsection*{Podtokovi}

\textbf{P1: Kreiranje vežbe}
\begin{enumerate}
    \item Trener bira opciju \textbf{Dodaj vežbu}.
    \item Sistem prikazuje formu za unos podataka o vežbi.
    \item Trener unosi naziv vežbe i učitava video snimak vežbe.
    \item Trener potvrđuje unos.
    \item Sistem validira podatke i čuva novu vežbu.
\end{enumerate}

\textbf{P2: Kreiranje treninga}
\begin{enumerate}
    \item Trener bira opciju \textbf{Dodaj trening}.
    \item Sistem proverava da li trener ima barem jednu sačuvanu vežbu.
    \item Sistem prikazuje formu za kreiranje treninga.
    \item Trener unosi naziv treninga i opis treninga.
    \item Sistem prikazuje listu vežbi koje je trener kreirao.
    \item Trener bira jednu ili više vežbi i za svaku definiše broj ponavljanja i broj setova.
    \item Trener potvrđuje kreiranje treninga.
    \item Sistem validira podatke i čuva novi trening.
\end{enumerate}

\textbf{P3: Brisanje vežbe}
\begin{enumerate}
    \item Trener bira opciju brisanja vežbe.
    \item Sistem proverava da li je vežba deo nekog postojećeg treninga.
    \item Ako vežba nije deo nijednog treninga, sistem briše vežbu.
    \item Ako je vežba deo jednog ili više treninga, sistem obaveštava trenera i može zahtevati dodatnu potvrdu ili zabraniti brisanje.
\end{enumerate}

\textbf{P4: Brisanje treninga}
\begin{enumerate}
    \item Trener bira opciju brisanja treninga.
    \item Sistem traži potvrdu brisanja.
    \item Nakon potvrde, sistem briše trening.
\end{enumerate}

\subsubsection*{Alternativni tokovi}

\textbf{A1: Neuspešan upload video snimka (podtok: P1 Kreiranje vežbe)}
\begin{itemize}
    \item U koraku 3 podtoka \textbf{Kreiranje vežbe} sistem ne uspeva da učita video snimak (npr. mrežna greška, prekid konekcije).
    \item Sistem obaveštava trenera da upload nije uspeo.
    \item Trener može pokušati ponovo da učita video (povratak na korak 3 podtoka) ili odustati, čime se podtok završava bez kreiranja vežbe.
\end{itemize}

\textbf{A2: Nevalidan format ili veličina video snimka (podtok: P1 Kreiranje vežbe)}
\begin{itemize}
    \item U koraku 4 podtoka \textbf{Kreiranje vežbe} sistem detektuje da video snimak ne ispunjava uslove (npr. nedozvoljen format ili prevelika veličina fajla).
    \item Sistem prikazuje poruku o grešci i traži da trener izabere drugi video snimak.
    \item Trener se vraća na korak 3 podtoka.
\end{itemize}

\textbf{A3: Trener pokušava da kreira trening bez postojećih vežbi (podtok: P2 Kreiranje treninga)}
\begin{itemize}
    \item U koraku 2 podtoka \textbf{Kreiranje treninga} sistem utvrđuje da trener nema nijednu sačuvanu vežbu.
    \item Sistem obaveštava trenera da je potrebno prvo kreirati barem jednu vežbu.
    \item Podtok se prekida, a trener se vraća na stranicu za upravljanje video treninzima.
\end{itemize}

\subsubsection*{Dodatne informacije}
\begin{itemize}
    \item Trener može kreirati proizvoljan broj vežbi pre kreiranja treninga.
    \item Trening mora sadržati barem jednu vežbu.
    \item Samo vežbe i treninzi koje je trener kreirao mogu biti menjani ili obrisani od strane tog trenera.
\end{itemize}

\begin{figure}[h!]
    \centering
    \includegraphics[width=0.9\linewidth]{images/upravljanje_video_treninzima.png}
    \caption{Dijagram slučaja upotrebe "Upravljanje video treninzima"}
\end{figure}

\begin{figure}[h!]
    \centering
    \includegraphics[width=0.9\linewidth]{diagrams/sequence_diagrams/manage_video_trainings.png}
    \caption{Sekvencijalni dijagram "Upravljanje video treninzima"}
\end{figure}

\FloatBarrier