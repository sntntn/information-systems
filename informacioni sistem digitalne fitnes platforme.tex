\section{Uvod}

Informacioni sistem digitalne fitnes platforme predstavlja sveobuhvatno softversko rešenje namenjeno podršci modernom načinu vežbanja, planiranja fizičkih aktivnosti i vođenja zdravog načina života. Sistem integriše više funkcionalnih celina, uključujući upravljanje treninzima (individualnim, grupnim i video treninzima), zakazivanje termina, komunikaciju između klijenata i trenera, izradu i praćenje planova ishrane, kao i obradu online plaćanja i slanje obaveštenja.

Cilj sistema je da omogući centralizovanu i efikasnu platformu koja povezuje klijente, trenere i administratore, olakšava organizaciju treninga i unapređuje korisničko iskustvo kroz digitalizaciju ključnih procesa. Sistem omogućava personalizovan pristup treniranju, praćenje napretka i kontinuiranu interakciju između učesnika, čime doprinosi boljoj motivaciji i kvalitetnijim rezultatima korisnika.


\subsection{Akteri}

Sistem prepoznaje sledeće grupe korisnika koji interaguju sa njegovim funkcionalnostima na različite načine:

\begin{itemize}
    \item \textbf{Klijent}: Registrovani korisnik sistema koji koristi usluge treninga. Klijent može pregledati dostupne treninge, rezervisati individualne i grupne treninge, kupovati i koristiti video treninge, komunicirati sa trenerima putem četa, vršiti online plaćanja i upravljati svojim obaveštenjima.
    
    \item \textbf{Trener}: Korisnik sistema zadužen za kreiranje i upravljanje treninzima. Trener može kreirati video vežbe i video treninge, definisati strukturu treninga, upravljati terminima, komunicirati sa klijentima i pratiti realizaciju treninga.

    \item \textbf{Administrator}: Korisnik sistema zadužen za tehničku administraciju i održavanje sistema. Administrator može dodavati ili uklanjati korisnike, upravljati pravima pristupa, kao i nadgledati rad sistema u celini.
    
    \item \textbf{Platni sistem}: Eksterni sistem koji obrađuje online plaćanja. Odgovoran je za validaciju i realizaciju transakcija prilikom kupovine treninga, rezervacije termina ili čet sesije.
\end{itemize}

\subsection{Korišćeni alati}

Za analizu, projektovanje i dokumentovanje informacionog sistema korišćen je skup standardizovanih jezika za modelovanje i odgovarajući softverski alati:

\begin{itemize}
    \item \textbf{UML (Unified Modeling Language)} dijagrami:
    \begin{itemize}
        \item \textbf{Slučajevi upotrebe}: Za opis funkcionalnosti sistema iz perspektive korisnika.
        \item \textbf{Dijagrami sekvence}: Za prikaz interakcije između aktera i sistemskih komponenti tokom realizacije slučajeva upotrebe.
        \item \textbf{Dijagrami aktivnosti}: Za modelovanje toka procesa i korisničkih radnji.
        \item \textbf{Dijagrami stanja}: Za prikaz životnog ciklusa važnih objekata u sistemu.
        \item \textbf{Dijagrami klasa}: Za prikaz modela baze podataka.
    \end{itemize}
    
    \item \textbf{BPMN (Business Process Model and Notation)}: Za modelovanje poslovnih procesa, naročito u delu obrade plaćanja i rezervacija.

    \item \textbf{DFD (Data Flow Diagram)}: Za prikaz toka podataka kroz sistem

    \item \textbf{LaTeX}: Korišćen za izradu i formatiranje tehničke dokumentacije, zbog podrške za strukturiranje sadržaja i profesionalni izgled dokumenata.

    \item \textbf{Vue.js}: Korišćen za izradu korisnickog interfejsa.

    \item \textbf{ScreenRec}: Korišćen za snimanje video zapisa korišćenja sistema.

\end{itemize}
\newpage
